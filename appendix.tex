\chapter{Framework}
\section{Configuration Parameters}
%\begin{table}
\begin{longtable}{%
|%
p{0.2\textwidth}%
|%
>{\raggedright\arraybackslash}%
p{0.135\textwidth}%
|%
>{\raggedright\arraybackslash}%
p{0.135\textwidth}%
|%
>{\raggedright\arraybackslash}%
p{0.4\textwidth}%
|%
}
\hline \textbf{Parameter} & \textbf{Possible values} & \textbf{Default} & \textbf{Description} \\ 
\hline
\hline class & A PASCAL class & bicycle & Specifies the class to restrict the bounding boxes to \\ 
\hline parts & $x \in \mathbb{N} \textbackslash 0$ & 4 & Number of parts per window \\ 
\hline clusters & $x \in \mathbb{N} \textbackslash 0$ & 1000 & Number of k-Means clusters to use \\ 
\hline integrals\allowbreak\_scale\allowbreak\_factor & $x \in \mathbb{R}, 0 < x \le 1$ & 1 & Specifies the percentage to which a integral image should be downsampled to \\ 
\hline stream\allowbreak\_max & $x \in \mathbb{N} \textbackslash 0$ & 100 & Maximum amount of images used from a PASCAL stream \\ 
\hline stream\allowbreak\_name & Any string & query & Name of the PASCAL stream to read images from \\ 
\hline codebook\allowbreak\_type & double, single & double & Internal datatype of the codebooks \\ 
\hline codebook\allowbreak\_scales\allowbreak\_count & $x \in \mathbb{N} \textbackslash 0$ & 3 & Amount of scale ranges in which integral images will be split into \\ 
\hline nonmax\allowbreak\_type\allowbreak\_min & true, false & true & Specifies if $\frac{\text{area}(A \cap B)}{\min(\text{area}(A), \text{area}(B))}$ or $\frac{\text{area}(A \cap B)}{\text{area}(A \cup B)}$ should be used for a non-maximum suppression \\ 
\hline use\allowbreak\_calibration & true, false & true & If true, a score calibration based on the query image is performed \\ 
\hline features\allowbreak\_per\allowbreak\_roi & $x \in \mathbb{R}, 0 < x$ & 2 & Amount of feature patches which have to exist per window axis \\ 
\hline query\allowbreak\_from\allowbreak\_integral & true, false & false & If true, the query codebooks will be retrieved from the image database instead of extracted from the query image \\ 
\hline default\allowbreak\_query\allowbreak\_file & An image id & 2008\allowbreak\_004363 & Used to automate the experiments instead of asking interactively for a query image \\ 
\hline default\allowbreak\_bounding\allowbreak\_box & [x, y, width, height] & Extracted from the PASCAL dataset & The bounding box of the query image part \\ 
\hline use\allowbreak\_libsvm\allowbreak\_classification & true, false & true & If true, the \verb|svmpredict| function is used to classify the database codebooks. Otherwise a custom implementation is used \\ 
\hline expand\allowbreak\_bboxes & true, false & true & Expands the bounding boxes by $\frac{1}{2}$ of the average feature patch size \\ 
\hline naiive\allowbreak\_integral\allowbreak\_backend & true, false & true & Specifies to use full integral images without any memory optimization techniques \\ 
\hline window\allowbreak\_margin & $x \in \mathbb{N} \textbackslash 0$ & 10 & Amount of pixels each window is shifted during the sliding window generation  \\ 
\hline max\allowbreak\_window\allowbreak\_scales & $x \in \mathbb{N} \textbackslash 0$ & 10 & Maximum amount of times a scaling is performed during the sliding window generation \\ 
\hline min\allowbreak\_window\allowbreak\_size & $x \in \mathbb{R}, 0 < x \le 1$ & 0.5 & Required percentage of a window area which has to be placed over an image (windows are getting clipped at the image boundaries)\\
\hline window\allowbreak\_generation\allowbreak\_relative\allowbreak\_move & $x \in \mathbb{R}$ & 0 & Percentage of its size a window should move during sliding window (0 means fixed window\allowbreak\_margin pixels) \\
\hline max\allowbreak\_window\allowbreak\_image\allowbreak\_ratio & $x \in \mathbb{R}, 0 < x \le 1$ & 1.0 & Maximum allowed size of a window in relation to the image \\
\hline use\allowbreak\_kdtree & true, false & false & Enables the kd-Tree storage backend \\ 
\hline integral\allowbreak\_backend\allowbreak\_overwrite & true, false & false & Enables the integral matrix reconstruction via overwriting the bottom right values \\ 
\hline integral\allowbreak\_backend\allowbreak\_sum & true, false & false & Enables the integral matrix reconstruction via summing up the bottom right values \\ 
\hline integral\allowbreak\_backend\allowbreak\_matlab\allowbreak\_sparse & true, false & false & Enables the \MATLAB sparse storage backend \\ 
\hline precalced\allowbreak\_windows & true, false & false & Disables integral images and precalculates codebooks in a sliding window manner \\ 
\hline inverse\allowbreak\_search & true, false & false & Use the information of integral images to reduce the amount of windows generated. \\
\hline memory\allowbreak\_cache & true, false & true & Reuse already loaded databases and negative codebooks (useful for development) \\
\hline use\allowbreak\_threading & true, false & true & Enables the use of the parallel toolbox \\
\hline window\allowbreak\_prefilter & true, false & false & Prefilter the windows even further to reduce the amount of searched windows \\
\hline fisher\allowbreak\_backend & true, false & false & Use fisher vectors instead of k-means clusters (experimental as it is not practical) \\
\hline
\caption{Configuration parameters for the part retrieval framework}
\label{tab:configuration_framework}
\end{longtable}
%\end{table}


%TODO API?
%\section{API}
%
\subsection{DEMO}

Demo implementation

\paragraph{Syntax:} \verb|demo(train, ...)|

Inputs:

\begin{tabular}{|l|p{5cm}|}
\hline
\textbf{Name} & \textbf{Description} \\
\hline \hline \\
train & boolean to indicate the creation of a image database. Default: false  \\ \hline
Name & Value pairs to override the default configuration  \\ \hline
\end{tabular}

\subsection{GENERATECLUSTER}

Generate a cluster model

\paragraph{Syntax:} \verb|cluster_model = generateCluster(params)|

Inputs:

\begin{tabular}{|l|p{5cm}|}
\hline
\textbf{Name} & \textbf{Description} \\
\hline \hline \\
params & Configuration parameters  \\ \hline
\end{tabular}
Outputs:

\begin{tabular}{|l|p{5cm}|}
\hline
\textbf{Name} & \textbf{Description} \\
\hline \hline \\
cluster\_model & The model  \\ \hline
\end{tabular}

\subsection{GETIMAGEDB}

Generate or load the image database

\paragraph{Syntax:} \verb|database = getImageDB(params, cluster_model)|

Inputs:

\begin{tabular}{|l|p{5cm}|}
\hline
\textbf{Name} & \textbf{Description} \\
\hline \hline \\
cluster\_model & The model for codebook generation  \\ \hline
params & Configuration parameters  \\ \hline
\end{tabular}
Outputs:

\begin{tabular}{|l|p{5cm}|}
\hline
\textbf{Name} & \textbf{Description} \\
\hline \hline \\
database & The database of integral images  \\ \hline
\end{tabular}

\subsection{GETSVM}

Train or load a exemplar SVM

\paragraph{Syntax:} \verb|svm_models = getSVM(params, cluster_model)|

Inputs:

\begin{tabular}{|l|p{5cm}|}
\hline
\textbf{Name} & \textbf{Description} \\
\hline \hline \\
cluster\_model & The model to generate codebooks  \\ \hline
parms & Configuration parameters  \\ \hline
\end{tabular}
Outputs:

\begin{tabular}{|l|p{5cm}|}
\hline
\textbf{Name} & \textbf{Description} \\
\hline \hline \\
svm\_models & SVM model struct  \\ \hline
\end{tabular}

\subsection{SEARCHINTERACTIVE}

Do a complete interactive database searchAsks the user for a image in the dataset image path.Allows to mark a ROI to search forStores results in the results/queries subdirectories

\paragraph{Syntax:} \verb|searchInteractive(params, cluster_model)|

Inputs:

\begin{tabular}{|l|p{5cm}|}
\hline
\textbf{Name} & \textbf{Description} \\
\hline \hline \\
cluster\_model & The model to generate codebooks  \\ \hline
params & Configuration parameters  \\ \hline
\end{tabular}

\subsection{SEARCHDATABASE}

Search the image database with the given SVM modelsStores results in the results/queries subdirectories

\paragraph{Syntax:} \verb|results = searchDatabase(params, database, svm_models, fit_params, roi)|

Inputs:

\begin{tabular}{|l|p{5cm}|}
\hline
\textbf{Name} & \textbf{Description} \\
\hline \hline \\
svm\_models & Struct array of svm models  \\ \hline
roi & The region of interest [x\_{min}, y\_{min}, x\_{max}, y\_{max}]  \\ \hline
params & Configuration parameters  \\ \hline
fit\_params & Gaussian curve 2D Vectors to adjust the SVM scores [$\mu$, $\sigma$]  \\ \hline
database & The image database as struct array of integral codebook images  \\ \hline
\end{tabular}
Outputs:

\begin{tabular}{|l|p{5cm}|}
\hline
\textbf{Name} & \textbf{Description} \\
\hline \hline \\
results & Cell array of results. Fields: curid, img, patch, score  \\ \hline
\end{tabular}

\subsection{SEARCHDATABASE}

Search the image database with the given SVM modelsStores results in the results/queries subdirectories

\paragraph{Syntax:} \verb|results = searchDatabase(params, database, svm_models, fit_params, roi)|

Inputs:

\begin{tabular}{|l|p{5cm}|}
\hline
\textbf{Name} & \textbf{Description} \\
\hline \hline \\
svm\_models & Struct array of svm models  \\ \hline
roi & The region of interest [x\_{min}, y\_{min}, x\_{max}, y\_{max}]  \\ \hline
params & Configuration parameters  \\ \hline
fit\_params & Gaussian curve 2D Vectors to adjust the SVM scores [$\mu$, $\sigma$]  \\ \hline
database & The image database as struct array of integral codebook images  \\ \hline
\end{tabular}
Outputs:

\begin{tabular}{|l|p{5cm}|}
\hline
\textbf{Name} & \textbf{Description} \\
\hline \hline \\
results & Cell array of results. Fields: curid, img, patch, score  \\ \hline
\end{tabular}

\subsection{CALIBRATE\_FIT}

Estimates the gaussian parameters for the given svm models

\paragraph{Syntax:} \verb|fit_params = calibrate_fit(params, svm_models, query_file, cluster_model)|

Inputs:

\begin{tabular}{|l|p{5cm}|}
\hline
\textbf{Name} & \textbf{Description} \\
\hline \hline \\
svm\_models & The SVM models  \\ \hline
cluster\_model & The model to generate the codebooks  \\ \hline
params & Configuration parameters  \\ \hline
query\_file & A pascal stream containing the file used to train the SVM  \\ \hline
\end{tabular}
Outputs:

\begin{tabular}{|l|p{5cm}|}
\hline
\textbf{Name} & \textbf{Description} \\
\hline \hline \\
fit\_params & Nx2 matrix of gaussian parameters. N: number of SVM models, Gaussian Parameters: [$\mu$, $\sigma$]  \\ \hline
\end{tabular}

\subsection{CALIBRATE\_RHO}

Tries to readjust the rho of the given SVM models to obtain better scores

\paragraph{Syntax:} \verb|rhos = calibrate_rho(params, svm_models, query_file, cluster_model, ground_truths)|

Inputs:

\begin{tabular}{|l|p{5cm}|}
\hline
\textbf{Name} & \textbf{Description} \\
\hline \hline \\
svm\_models & The SVM models  \\ \hline
cluster\_model & The model to generate the codebooks  \\ \hline
ground\_truths & Nx4 Matrix of bounding boxes inside the query image (e.g. the ROI)  \\ \hline
params & Configuration parameters  \\ \hline
query\_file & A pascal stream containing the file used to train the SVM  \\ \hline
\end{tabular}
Outputs:

\begin{tabular}{|l|p{5cm}|}
\hline
\textbf{Name} & \textbf{Description} \\
\hline \hline \\
rhos & Mx1 vector of new rhos. M: number of SVM models  \\ \hline
\end{tabular}

\subsection{CONVERT\_SERIALIZE}

Creates a corresponding file with toggled serialization

\paragraph{Syntax:} \verb|convert_serialize(filename)|

Inputs:

\begin{tabular}{|l|p{5cm}|}
\hline
\textbf{Name} & \textbf{Description} \\
\hline \hline \\
filename & The file to invert  \\ \hline
\end{tabular}

\subsection{GET\_DEFAULT\_CONFIGURATION}

returns a default configuration

\paragraph{Syntax:} \verb|params = get_default_configuration()|

Outputs:

\begin{tabular}{|l|p{5cm}|}
\hline
\textbf{Name} & \textbf{Description} \\
\hline \hline \\
params & The default configuration  \\ \hline
\end{tabular}

\subsection{GET\_CODEBOOKS}

Get integral codebooks from given features

\paragraph{Syntax:} \verb|codebooks = get_codebooks(params, features, cluster_model)|

Inputs:

\begin{tabular}{|l|p{5cm}|}
\hline
\textbf{Name} & \textbf{Description} \\
\hline \hline \\
features & A feature struct array. Required Fields: curid, X, bbs, window2feature  \\ \hline
cluster\_model & A model from get\_cluster  \\ \hline
params & Configuration struct  \\ \hline
\end{tabular}
Outputs:

\begin{tabular}{|l|p{5cm}|}
\hline
\textbf{Name} & \textbf{Description} \\
\hline \hline \\
codebooks & A struct array with fields: I, curid, size  \\ \hline
\end{tabular}

\subsection{GET\_CODEBOOK\_INTEGRALS}

Get integral codebooks from given features

\paragraph{Syntax:} \verb|integrals = get_codebook_integrals(params, features, cluster_model, roi_size)|

Inputs:

\begin{tabular}{|l|p{5cm}|}
\hline
\textbf{Name} & \textbf{Description} \\
\hline \hline \\
features & A feature struct array. Required Fields: curid, X, bbs, I\_size, scales  \\ \hline
cluster\_model & A model from get\_cluster  \\ \hline
params & Configuration struct  \\ \hline
roi\_size & Size of the query part  \\ \hline
\end{tabular}
Outputs:

\begin{tabular}{|l|p{5cm}|}
\hline
\textbf{Name} & \textbf{Description} \\
\hline \hline \\
integrals & A struct array with fields: I, I\_size, curid, scale\_factor,max\_size, min\_size, tree,scores, idx, coords  \\ \hline
\end{tabular}

\subsection{GET\_CACHE\_BASEDIR}

Gets and creates the cache dir

\paragraph{Syntax:} \verb|basedir = get_cache_basedir(params, create_dir)|

Inputs:

\begin{tabular}{|l|p{5cm}|}
\hline
\textbf{Name} & \textbf{Description} \\
\hline \hline \\
create\_dir & Boolean to indicate the automatic creation of the dir  \\ \hline
params & Configuration struct  \\ \hline
\end{tabular}
Outputs:

\begin{tabular}{|l|p{5cm}|}
\hline
\textbf{Name} & \textbf{Description} \\
\hline \hline \\
basedir & The cache dir  \\ \hline
\end{tabular}

\subsection{GET\_CACHE\_NAME}

Gets the cache filename

\paragraph{Syntax:} \verb|cachename = get_cache_name(params, roi_size, create_dir)|

Inputs:

\begin{tabular}{|l|p{5cm}|}
\hline
\textbf{Name} & \textbf{Description} \\
\hline \hline \\
create\_dir & Boolean to indicate the automatic creation of the dir  \\ \hline
params & Configuration struct  \\ \hline
roi\_size & Vector containing the size of the roi or empty  \\ \hline
\end{tabular}
Outputs:

\begin{tabular}{|l|p{5cm}|}
\hline
\textbf{Name} & \textbf{Description} \\
\hline \hline \\
cachename & The cache file name  \\ \hline
\end{tabular}

\subsection{GET\_IMG\_CACHE\_NAME}

Gets the cache filename of a single image

\paragraph{Syntax:} \verb|imgcachename = get_img_cache_name(params, roi_size, create_dir)|

Inputs:

\begin{tabular}{|l|p{5cm}|}
\hline
\textbf{Name} & \textbf{Description} \\
\hline \hline \\
create\_dir & Boolean to indicate the automatic creation of the dir  \\ \hline
params & Configuration struct  \\ \hline
roi\_size & Vector containing the size of the roi or empty  \\ \hline
\end{tabular}
Outputs:

\begin{tabular}{|l|p{5cm}|}
\hline
\textbf{Name} & \textbf{Description} \\
\hline \hline \\
imgcachename & The cache file name  \\ \hline
\end{tabular}

\subsection{GET\_CLUSTER}

Clusters the given features with kmeans

\paragraph{Syntax:} \verb|model = get_cluster( params, features )|

Inputs:

\begin{tabular}{|l|p{5cm}|}
\hline
\textbf{Name} & \textbf{Description} \\
\hline \hline \\
features & The feature struct array (Required Fields: X)  \\ \hline
params & The configuration struct used for caching and profiling  \\ \hline
\end{tabular}
Outputs:

\begin{tabular}{|l|p{5cm}|}
\hline
\textbf{Name} & \textbf{Description} \\
\hline \hline \\
model & The computed model (Fields: centroids. Methods: feature2codebook(params, feature), feature2codebookintegral(params, feature))  \\ \hline
\end{tabular}

\subsection{FEATURE2CODEBOOK}

Calculates a codebook from a given feature struct

\paragraph{Syntax:} \verb|codebook = feature2codebook(params, feature, model)|

Inputs:

\begin{tabular}{|l|p{5cm}|}
\hline
\textbf{Name} & \textbf{Description} \\
\hline \hline \\
model & The cluster model. Required fields: centroids  \\ \hline
feature & The feature struct. Required fields: X, window2feature, bbs  \\ \hline
params & The configuration struct. Required fields: parts, codebook\_type, profile (if profiling is required)  \\ \hline
\end{tabular}
Outputs:

\begin{tabular}{|l|p{5cm}|}
\hline
\textbf{Name} & \textbf{Description} \\
\hline \hline \\
codebook & A NxM matrix. N: params.parts * size(centroids, 1), M: length(window2feature)  \\ \hline
\end{tabular}

\subsection{FEATURE2CODEBOOK}

Calculates a codebook integral from a given feature struct

\paragraph{Syntax:} \verb|codebook = feature2codebookintegral(params, feature, model)|

Inputs:

\begin{tabular}{|l|p{5cm}|}
\hline
\textbf{Name} & \textbf{Description} \\
\hline \hline \\
model & The cluster model. Required fields: centroids  \\ \hline
feature & The feature struct. Required fields: X, bbs, I\_size, scales  \\ \hline
params & The configuration struct. Required fields: codebook\_type, profile (if profiling is required)  \\ \hline
\end{tabular}
Outputs:

\begin{tabular}{|l|p{5cm}|}
\hline
\textbf{Name} & \textbf{Description} \\
\hline \hline \\
scales & A Cell of size S containing the associated scales.  \\ \hline
codebook & A SxNxWxH matrix. S: different scales, N: size(centroids, 1), W: I\_size(2), H: I\_size(1)  \\ \hline
\end{tabular}

\subsection{FEATURE2CODEBOOK}

Calculates a codebook integral from a given feature struct

\paragraph{Syntax:} \verb|codebook = feature2codebookintegral(params, feature, model)|

Inputs:

\begin{tabular}{|l|p{5cm}|}
\hline
\textbf{Name} & \textbf{Description} \\
\hline \hline \\
model & The cluster model. Required fields: centroids  \\ \hline
feature & The feature struct. Required fields: X, bbs, I\_size, scales  \\ \hline
params & The configuration struct. Required fields: codebook\_type, profile (if profiling is required)  \\ \hline
\end{tabular}
Outputs:

\begin{tabular}{|l|p{5cm}|}
\hline
\textbf{Name} & \textbf{Description} \\
\hline \hline \\
scales & A Cell of size S containing the associated scales.  \\ \hline
codebook & A SxNxWxH matrix. S: different scales, N: size(centroids, 1), W: I\_size(2), H: I\_size(1)  \\ \hline
\end{tabular}

\subsection{GET\_CODEBOOK\_FROM\_WINDOWS}

Get images codebooks from given features

\paragraph{Syntax:} \verb|images = get_codebook_from_windows(params, features, cluster_model, roi_size)|

Inputs:

\begin{tabular}{|l|p{5cm}|}
\hline
\textbf{Name} & \textbf{Description} \\
\hline \hline \\
features & A feature struct array. Required Fields: curid, X, bbs, I\_size, scales  \\ \hline
cluster\_model & A model from get\_cluster  \\ \hline
params & Configuration struct  \\ \hline
roi\_size & Size of the query part  \\ \hline
\end{tabular}
Outputs:

\begin{tabular}{|l|p{5cm}|}
\hline
\textbf{Name} & \textbf{Description} \\
\hline \hline \\
images & A struct array with fields: curid, scale\_factor, max\_size,min\_size, bboxes, codebooks, images  \\ \hline
\end{tabular}

\subsection{GET\_CACHE\_BASEDIR}

Gets and creates the cache dir

\paragraph{Syntax:} \verb|basedir = get_cache_basedir(params, create_dir)|

Inputs:

\begin{tabular}{|l|p{5cm}|}
\hline
\textbf{Name} & \textbf{Description} \\
\hline \hline \\
create\_dir & Boolean to indicate the automatic creation of the dir  \\ \hline
params & Configuration struct  \\ \hline
\end{tabular}
Outputs:

\begin{tabular}{|l|p{5cm}|}
\hline
\textbf{Name} & \textbf{Description} \\
\hline \hline \\
basedir & The cache dir  \\ \hline
\end{tabular}

\subsection{GET\_CACHE\_NAME}

Gets the cache filename

\paragraph{Syntax:} \verb|cachename = get_cache_name(params, roi_size, create_dir)|

Inputs:

\begin{tabular}{|l|p{5cm}|}
\hline
\textbf{Name} & \textbf{Description} \\
\hline \hline \\
create\_dir & Boolean to indicate the automatic creation of the dir  \\ \hline
params & Configuration struct  \\ \hline
roi\_size & Vector containing the size of the roi or empty  \\ \hline
\end{tabular}
Outputs:

\begin{tabular}{|l|p{5cm}|}
\hline
\textbf{Name} & \textbf{Description} \\
\hline \hline \\
cachename & The cache file name  \\ \hline
\end{tabular}

\subsection{GET\_IMG\_CACHE\_NAME}

Gets the cache filename of a single image

\paragraph{Syntax:} \verb|imgcachename = get_img_cache_name(params, roi_size, create_dir)|

Inputs:

\begin{tabular}{|l|p{5cm}|}
\hline
\textbf{Name} & \textbf{Description} \\
\hline \hline \\
create\_dir & Boolean to indicate the automatic creation of the dir  \\ \hline
params & Configuration struct  \\ \hline
roi\_size & Vector containing the size of the roi or empty  \\ \hline
\end{tabular}
Outputs:

\begin{tabular}{|l|p{5cm}|}
\hline
\textbf{Name} & \textbf{Description} \\
\hline \hline \\
imgcachename & The cache file name  \\ \hline
\end{tabular}

\subsection{GET\_IMAGE}

Loads a pascal image by its name

\paragraph{Syntax:} \verb|I = get_image( params, name )|

Inputs:

\begin{tabular}{|l|p{5cm}|}
\hline
\textbf{Name} & \textbf{Description} \\
\hline \hline \\
params & Configuration struct  \\ \hline
name & The image name  \\ \hline
\end{tabular}
Outputs:

\begin{tabular}{|l|p{5cm}|}
\hline
\textbf{Name} & \textbf{Description} \\
\hline \hline \\
I & The loaded image  \\ \hline
\end{tabular}

\subsection{GET\_FEATURES\_FROM\_STREAM}

Calculates HoG features from a given Pascal Stream

\paragraph{Syntax:} \verb|features = get_features_from_stream( params, stream )|

Inputs:

\begin{tabular}{|l|p{5cm}|}
\hline
\textbf{Name} & \textbf{Description} \\
\hline \hline \\
stream & A Pascal stream  \\ \hline
params & Configuration struct  \\ \hline
\end{tabular}
Outputs:

\begin{tabular}{|l|p{5cm}|}
\hline
\textbf{Name} & \textbf{Description} \\
\hline \hline \\
features & A feature struct array. Fields: curid, objectid,feature\_type, I\_size, X, M, scales, bbs, window2feature,area, all\_scales  \\ \hline
\end{tabular}

\subsection{EXTRACT\_QUERY\_CODEBOOK}

extracts a query codebook based on the configuration

\paragraph{Syntax:} \verb|query_codebooks = extract_query_codebook( params, cluster_model, query_file, roi_size )|

Inputs:

\begin{tabular}{|l|p{5cm}|}
\hline
\textbf{Name} & \textbf{Description} \\
\hline \hline \\
cluster\_model & Model representing the current clustering method  \\ \hline
params & Configuration struct  \\ \hline
roi\_size & Bounding box of the query part  \\ \hline
query\_file & Struct with information about the query file  \\ \hline
\end{tabular}
Outputs:

\begin{tabular}{|l|p{5cm}|}
\hline
\textbf{Name} & \textbf{Description} \\
\hline \hline \\
query\_codebooks & the resulting codebook  \\ \hline
\end{tabular}

\subsection{PREPARE\_FEATURES}

load or generate features for the given stream set

\paragraph{Syntax:} \verb|features = prepare_features(params, query_stream_set)|

Inputs:

\begin{tabular}{|l|p{5cm}|}
\hline
\textbf{Name} & \textbf{Description} \\
\hline \hline \\
query\_stream\_set & stream set containing the files to load  \\ \hline
params & Configuration struct  \\ \hline
\end{tabular}
Outputs:

\begin{tabular}{|l|p{5cm}|}
\hline
\textbf{Name} & \textbf{Description} \\
\hline \hline \\
features & struct array of features  \\ \hline
\end{tabular}

\subsection{GET\_FULL\_NEG\_MODEL}

Loads the negative model for whitened HoGs

\paragraph{Syntax:} \verb|neg_model = get_full_neg_model()|

Outputs:

\begin{tabular}{|l|p{5cm}|}
\hline
\textbf{Name} & \textbf{Description} \\
\hline \hline \\
neg\_model & The negative model  \\ \hline
\end{tabular}

\subsection{GET\_SVMS}

Get trained exemplar svms from given codebooks

\paragraph{Syntax:} \verb|svm_models = get_svms( params, query_codebooks, neg_codebooks )|

Inputs:

\begin{tabular}{|l|p{5cm}|}
\hline
\textbf{Name} & \textbf{Description} \\
\hline \hline \\
neg\_codebooks & Codebook struct array (Fields: I) or NxM matrix. N: Num codebooks, M: Codebook size  \\ \hline
params & Configuration struct  \\ \hline
query\_codebooks & A codebook struct array. Required Fields: I, size, curid  \\ \hline
\end{tabular}
Outputs:

\begin{tabular}{|l|p{5cm}|}
\hline
\textbf{Name} & \textbf{Description} \\
\hline \hline \\
svm\_models & Struct array of svm models. Fields: cb\_size, codebook, curid, model  \\ \hline
\end{tabular}

\subsection{ADJUST\_SCORES}

adjusts scores based on given gaussian fit parametersScores will applied on a gaussian curve. The curve is shifted by $3 * \sigma$and the resulting scores rescaled by $new\_scores * 2 - 1$

\paragraph{Syntax:} \verb|[ new_scores ] = adjust_scores( params, fit_params, scores )|

Inputs:

\begin{tabular}{|l|p{5cm}|}
\hline
\textbf{Name} & \textbf{Description} \\
\hline \hline \\
scores & Score vector to adjust  \\ \hline
params & Configuration parameters as struct  \\ \hline
fit\_params & 2D-Vector. [$\mu$, $\sigma$]  \\ \hline
\end{tabular}
Outputs:

\begin{tabular}{|l|p{5cm}|}
\hline
\textbf{Name} & \textbf{Description} \\
\hline \hline \\
new\_scores & Adjusted scores  \\ \hline
\end{tabular}

\subsection{GET\_RANGE\_FOR\_SCALE}

get the range of given scales

\paragraph{Syntax:} \verb|[ min_size, max_size ] = get_range_for_scale(params, current_scales)|

Inputs:

\begin{tabular}{|l|p{5cm}|}
\hline
\textbf{Name} & \textbf{Description} \\
\hline \hline \\
current\_scales & NxM matrix of scales  \\ \hline
params & Configuration struct  \\ \hline
\end{tabular}
Outputs:

\begin{tabular}{|l|p{5cm}|}
\hline
\textbf{Name} & \textbf{Description} \\
\hline \hline \\
max\_size & vector with N maximum values  \\ \hline
min\_size & vector with N minimum values  \\ \hline
\end{tabular}

\subsection{FILTER\_FEATURES}

Filters given featuresRemoves features with too little texture or which are too close to the negative mode of whitened features

\paragraph{Syntax:} \verb|features = filter_features(params, features)|

Inputs:

\begin{tabular}{|l|p{5cm}|}
\hline
\textbf{Name} & \textbf{Description} \\
\hline \hline \\
features & The feature struct array (Fields: X, M, distVec, scales, bbs, window2feature)  \\ \hline
params & The configuration struct. Used for profiling and caching  \\ \hline
\end{tabular}
Outputs:

\begin{tabular}{|l|p{5cm}|}
\hline
\textbf{Name} & \textbf{Description} \\
\hline \hline \\
features & The filtered feature struct array with new logical vector deletedFeatures  \\ \hline
\end{tabular}

\subsection{CALC\_CODEBOOKS}

Extracts codebooks and bounding boxes from a given image database

\paragraph{Syntax:} \verb|[ bboxes, codebooks, images ] = calc_codebooks(params, database, windows_bb, num_parts )|

Inputs:

\begin{tabular}{|l|p{5cm}|}
\hline
\textbf{Name} & \textbf{Description} \\
\hline \hline \\
windows\_bb & A Nx4 matrix of bounding boxes to to extract ($x\_{min}$, $y\_{min}$, $x\_{max}$, $y\_{max}$)  \\ \hline
num\_parts & (Even) number of segments a window should be divided to  \\ \hline
params & The configuration parameters, currently only required for profiling  \\ \hline
database & The image database as struct array with the fields I, curid and optionally scale\_factor  \\ \hline
\end{tabular}
Outputs:

\begin{tabular}{|l|p{5cm}|}
\hline
\textbf{Name} & \textbf{Description} \\
\hline \hline \\
images & A 1xN dimensional index vector for assigning codebooks to images  \\ \hline
bboxes & A Nx4 matrix of bounding boxes related to the extracted codebooks ($x\_{min}$, $y\_{min}$, $x\_{max}$, $y\_{max}$)  \\ \hline
codebooks & A Nx(M*num\_parts) matrix of M dimensional codebooks  \\ \hline
\end{tabular}

\subsection{EXPECTEDCODEBOOKS}

Tries to estimate the amount of codebooks extractedInternally used to preallocate the memory for speed

\paragraph{Syntax:} \verb|[expectedCodebookCount, codebookSize] = expectedCodebooks(database, windows_bb, num_parts)|

Inputs:

\begin{tabular}{|l|p{5cm}|}
\hline
\textbf{Name} & \textbf{Description} \\
\hline \hline \\
windows\_bb & A Nx4 matrix of bounding boxes to to extract  \\ \hline
num\_parts & (Even) number of segments a window should be divided to  \\ \hline
database & The image database as struct array with the field I  \\ \hline
\end{tabular}
Outputs:

\begin{tabular}{|l|p{5cm}|}
\hline
\textbf{Name} & \textbf{Description} \\
\hline \hline \\
expectedCodebookCount & Estimated number of codebooks which will be extracted  \\ \hline
codebookSize & Size of the resulting codebooks (M*num\_parts)  \\ \hline
\end{tabular}

\subsection{IS\_IN\_SCALE}

Tests if the requested sizes are within the a given scale

\paragraph{Syntax:} \verb|in_scale = is_in_scale(params, min_size, max_size, requested_size)|

Inputs:

\begin{tabular}{|l|p{5cm}|}
\hline
\textbf{Name} & \textbf{Description} \\
\hline \hline \\
max\_size & upper bound (or [] if lower bound is requested)  \\ \hline
params & Configuration struct  \\ \hline
min\_size & lower bound (or [] if upper bound is requested)  \\ \hline
\end{tabular}
Outputs:

\begin{tabular}{|l|p{5cm}|}
\hline
\textbf{Name} & \textbf{Description} \\
\hline \hline \\
in\_scale & logic vector  \\ \hline
\end{tabular}

\subsection{CALC\_WINDOWS}

Calculates the windows wich have to be extracted in a sliding window approach

\paragraph{Syntax:} \verb|windows_bb = calc_windows(params, w, h, cbw, cbh, I )|

Inputs:

\begin{tabular}{|l|p{5cm}|}
\hline
\textbf{Name} & \textbf{Description} \\
\hline \hline \\
cbw & Smallest width of a window  \\ \hline
cbh & Smallest height of a window  \\ \hline
I & Optional test image to produce a intagral.avi file for visualization  \\ \hline
w & Width of the image  \\ \hline
params & Configuration struct  \\ \hline
h & Height of the image  \\ \hline
\end{tabular}
Outputs:

\begin{tabular}{|l|p{5cm}|}
\hline
\textbf{Name} & \textbf{Description} \\
\hline \hline \\
windows\_bb & Nx4 matrix of windows ($x\_{min}$, $y\_{min}$, $x\_{max}$, $y\_{max}$)  \\ \hline
\end{tabular}

\subsection{SPARSE\_CODEBOOK}

Matlab implementation of the reconstruction of a kd-Tree integral image

\paragraph{Syntax:} \verb|codebook = sparse_codebook(integral, x, y)|

Inputs:

\begin{tabular}{|l|p{5cm}|}
\hline
\textbf{Name} & \textbf{Description} \\
\hline \hline \\
x, y & Coordinate of requested codebook  \\ \hline
integral & An integral image with kd-Tree storage backend  \\ \hline
\end{tabular}
Outputs:

\begin{tabular}{|l|p{5cm}|}
\hline
\textbf{Name} & \textbf{Description} \\
\hline \hline \\
codebook & the reconstructed codebook  \\ \hline
\end{tabular}

\subsection{GETCODEBOOKSFROMINTEGRAL}

Extract codebooks from a single integral image

\paragraph{Syntax:} \verb|codebooks = getCodebooksFromIntegral(params, integral_img, bboxes, num_parts)|

Inputs:

\begin{tabular}{|l|p{5cm}|}
\hline
\textbf{Name} & \textbf{Description} \\
\hline \hline \\
bboxes & Mx4 Matrix of bounding boxes to extract  \\ \hline
num\_parts & (Even) number of segments to divide a single window into  \\ \hline
integral\_img & SxNxWxH Matrix. S: scales, N: Codebook Size, W: Width, H: Height  \\ \hline
params & Configuration struct  \\ \hline
\end{tabular}
Outputs:

\begin{tabular}{|l|p{5cm}|}
\hline
\textbf{Name} & \textbf{Description} \\
\hline \hline \\
codebooks & N2xSxM Matrix. N2: N*num\_parts  \\ \hline
\end{tabular}

\subsection{ESTIMATE\_FIT\_PARAMS}

Estimates parameters of a gaussian curveCan be used to adjust the scores

\paragraph{Syntax:} \verb|fit_params = estimate_fit_params( params, scores )|

Inputs:

\begin{tabular}{|l|p{5cm}|}
\hline
\textbf{Name} & \textbf{Description} \\
\hline \hline \\
scores & Score vector to fit to  \\ \hline
params & Configuration struct (currently not used)  \\ \hline
\end{tabular}
Outputs:

\begin{tabular}{|l|p{5cm}|}
\hline
\textbf{Name} & \textbf{Description} \\
\hline \hline \\
fit\_params & 2D-Vector of parameters [$\mu$, $\sigma$]  \\ \hline
\end{tabular}

\subsection{GETPARTS}

Calculates the segments of a window

\paragraph{Syntax:} \verb|[ xsteps, ysteps ] = getParts( minX, minY, maxX, maxY, num_parts)|

Inputs:

\begin{tabular}{|l|p{5cm}|}
\hline
\textbf{Name} & \textbf{Description} \\
\hline \hline \\
num\_parts & (Even) number of segments  \\ \hline
minY & Lowest y value  \\ \hline
maxY & Highest y value  \\ \hline
minX & Lowest x value  \\ \hline
maxX & Highest x value  \\ \hline
\end{tabular}
Outputs:

\begin{tabular}{|l|p{5cm}|}
\hline
\textbf{Name} & \textbf{Description} \\
\hline \hline \\
ysteps & Y offsets inside the bounding box [[from; to], ...] (2xnum\_parts Matrix)  \\ \hline
xsteps & X offsets inside the bounding box [[from; to], ...] (2xnum\_parts Matrix)  \\ \hline
\end{tabular}

\subsection{JOIN\_PART\_COVARIANCES}

Joins multiple small files into one matrix

\paragraph{Syntax:} \verb|neg_model = join_part_covariances(num_parts)|

Inputs:

\begin{tabular}{|l|p{5cm}|}
\hline
\textbf{Name} & \textbf{Description} \\
\hline \hline \\
num\_parts & Number of files to join  \\ \hline
\end{tabular}
Outputs:

\begin{tabular}{|l|p{5cm}|}
\hline
\textbf{Name} & \textbf{Description} \\
\hline \hline \\
neg\_model & the negative model for whitened HoGs  \\ \hline
\end{tabular}

\subsection{REDUCE\_MATCHES}

Reduces the amount of detected matches with a non-max suppression

\paragraph{Syntax:} \verb|[bbox, scores, idx] = reduce_matches(params, bbox, scores)|

Inputs:

\begin{tabular}{|l|p{5cm}|}
\hline
\textbf{Name} & \textbf{Description} \\
\hline \hline \\
bbox & Nx4 matrix of bounding boxes [x, y, w, h]  \\ \hline
params & Configuration parameters  \\ \hline
scores & N dimensional vector of scores  \\ \hline
\end{tabular}
Outputs:

\begin{tabular}{|l|p{5cm}|}
\hline
\textbf{Name} & \textbf{Description} \\
\hline \hline \\
bbox & New bounding boxes  \\ \hline
idx & Mapping between input and output (index vector)  \\ \hline
scores & New scores  \\ \hline
\end{tabular}

\subsection{GETHOGSINSIDEBOX}

Restrict calculated HoGs to a given mask

\paragraph{Syntax:} \verb|[X, W, M, offsets, uus, vvs, scales] = getHogsInsideBox(t, I, mask, params)|

Inputs:

\begin{tabular}{|l|p{5cm}|}
\hline
\textbf{Name} & \textbf{Description} \\
\hline \hline \\
I & Source image of features  \\ \hline
mask & logical mask to restrict features (same size as source image)  \\ \hline
t & Feature pyramid  \\ \hline
params & Configuration parameters  \\ \hline
\end{tabular}
Outputs:

\begin{tabular}{|l|p{5cm}|}
\hline
\textbf{Name} & \textbf{Description} \\
\hline \hline \\
W & wavelet  \\ \hline
uus & Patch positions  \\ \hline
offsets & Patch offsets  \\ \hline
M & Gradient  \\ \hline
vvs & Patch positions  \\ \hline
X & Features (Nx775)  \\ \hline
scales & Patch scales  \\ \hline
\end{tabular}

\subsection{WHITEN\_FEATURES}

Transforms HoG features into whitened HoGs

\paragraph{Syntax:} \verb|features = whiten_features( params, features )|

Inputs:

\begin{tabular}{|l|p{5cm}|}
\hline
\textbf{Name} & \textbf{Description} \\
\hline \hline \\
features & Feature struct array. Required Fields: X, M  \\ \hline
params & Configuration struct  \\ \hline
\end{tabular}
Outputs:

\begin{tabular}{|l|p{5cm}|}
\hline
\textbf{Name} & \textbf{Description} \\
\hline \hline \\
features & Whitened feature struct array. New Field: distVec  \\ \hline
\end{tabular}

\subsection{MERGE\_STRUCTS}

Merges multiple structs togetherFields of the first struct gets overriden by the second, the third, ....

\paragraph{Syntax:} \verb|result = merge_structs(struct1, struct2, ...)|

Outputs:

\begin{tabular}{|l|p{5cm}|}
\hline
\textbf{Name} & \textbf{Description} \\
\hline \hline \\
result & The merged struct  \\ \hline
\end{tabular}

\subsection{PROFILE\_STEP}

Logs an execution traceLogs where it was called and which time elapsed since the profiling start

\paragraph{Syntax:} \verb|params = profile_log( params )|

Inputs:

\begin{tabular}{|l|p{5cm}|}
\hline
\textbf{Name} & \textbf{Description} \\
\hline \hline \\
params & A configuration struct produced by profile\_start  \\ \hline
\end{tabular}
Outputs:

\begin{tabular}{|l|p{5cm}|}
\hline
\textbf{Name} & \textbf{Description} \\
\hline \hline \\
params & The input struct with updated profile.steps field (not required for subsequential calls)  \\ \hline
\end{tabular}

\subsection{MAKE\_SOUND}

Plays a soundtrack to get attentionCould be used to signal the end of a computation or the presence of an error

\paragraph{Syntax:} \verb|make_sound( finished )|

Inputs:

\begin{tabular}{|l|p{5cm}|}
\hline
\textbf{Name} & \textbf{Description} \\
\hline \hline \\
finished & Boolean, specifies if a gong (false) or a handel (true) should be played  \\ \hline
\end{tabular}

\subsection{SAVE\_EX}

Advanced wrapper around saveProvides status information and to serialize the variables with hlp\_serialize

\paragraph{Syntax:} \verb|save_ex(filename, save_args, ...)|

Inputs:

\begin{tabular}{|l|p{5cm}|}
\hline
\textbf{Name} & \textbf{Description} \\
\hline \hline \\
save\_args & Variadic arguments for matlabs save function  \\ \hline
filename & The file to save to  \\ \hline
\end{tabular}

\subsection{PARSE\_KEYWORDS}

Parses a list of arguments into a structRequires arguments of the form Keyword1, Value1, Keyword2, Value2, ...

\paragraph{Syntax:} \verb|keywords = parse_keywords(input_args, allowed_keywords)|

Inputs:

\begin{tabular}{|l|p{5cm}|}
\hline
\textbf{Name} & \textbf{Description} \\
\hline \hline \\
input\_args & Cell array of input arguments (e.g. varargin)  \\ \hline
allowed\_keywords & Optional cell array of allowed keywords  \\ \hline
\end{tabular}
Outputs:

\begin{tabular}{|l|p{5cm}|}
\hline
\textbf{Name} & \textbf{Description} \\
\hline \hline \\
keywords & Struct of keyword, value pairs  \\ \hline
\end{tabular}

\subsection{LOAD\_EX}

Advanced load wrapperPrints status information and allows to load files serialized with hlp\_deserializeBehaves exactly like matlabs load function

\paragraph{Syntax:} \verb|out = load_ex(filename, load_arg, ...)|

Inputs:

\begin{tabular}{|l|p{5cm}|}
\hline
\textbf{Name} & \textbf{Description} \\
\hline \hline \\
load\_arg & Optional, variadic arguments for matlabs load function  \\ \hline
filename & File to load  \\ \hline
\end{tabular}
Outputs:

\begin{tabular}{|l|p{5cm}|}
\hline
\textbf{Name} & \textbf{Description} \\
\hline \hline \\
serialized & optional boolean indicating the load of a serialized var  \\ \hline
out & optional struct containing the loaded variables  \\ \hline
\end{tabular}

\subsection{STRUCT2STR}

converts a struct into a printable string

\paragraph{Syntax:} \verb|str = struct2str(in, recursive)|

Inputs:

\begin{tabular}{|l|p{5cm}|}
\hline
\textbf{Name} & \textbf{Description} \\
\hline \hline \\
recursive & boolean to indicate a recursive print  \\ \hline
in & the struct to print  \\ \hline
\end{tabular}
Outputs:

\begin{tabular}{|l|p{5cm}|}
\hline
\textbf{Name} & \textbf{Description} \\
\hline \hline \\
str & String containing a text representation of the struct  \\ \hline
\end{tabular}

\subsection{PROFILE\_STOP}

Stops the profilingRecords the total execution time during the profiling

\paragraph{Syntax:} \verb|profile_stop( params )|

Inputs:

\begin{tabular}{|l|p{5cm}|}
\hline
\textbf{Name} & \textbf{Description} \\
\hline \hline \\
params & The configuration struct processed by profile\_start  \\ \hline
\end{tabular}

\subsection{PROFILE\_START}

Starts an execution traceupdates the given configuration struct by adding a profile field with subfieldsrecords the start time.

\paragraph{Syntax:} \verb|params = profile_start( params )|

Inputs:

\begin{tabular}{|l|p{5cm}|}
\hline
\textbf{Name} & \textbf{Description} \\
\hline \hline \\
params & A configuration struct with a configured dataset (dataset.localdir field is required)  \\ \hline
\end{tabular}
Outputs:

\begin{tabular}{|l|p{5cm}|}
\hline
\textbf{Name} & \textbf{Description} \\
\hline \hline \\
params & The updated struct. Required for profile\_log and profile\_stop calls  \\ \hline
\end{tabular}

\subsection{ALLOC\_STRUCT\_ARRAY}

Allocates a struct array of given size with given fieldsSorts fieldnames to be in line with matlabs save -struct

\paragraph{Syntax:} \verb|array = alloc_struct_array( size, field, ... )|

Inputs:

\begin{tabular}{|l|p{5cm}|}
\hline
\textbf{Name} & \textbf{Description} \\
\hline \hline \\
field & Variable number of fields to be contained in the array  \\ \hline
size & The requested size of the struct array (vector possible)  \\ \hline
\end{tabular}
Outputs:

\begin{tabular}{|l|p{5cm}|}
\hline
\textbf{Name} & \textbf{Description} \\
\hline \hline \\
array & The 1xsize struct array  \\ \hline
\end{tabular}

\subsection{SET\_LOG\_LEVEL}

set current log level

\paragraph{Syntax:} \verb|set_log_level(lvl)|

Inputs:

\begin{tabular}{|l|p{5cm}|}
\hline
\textbf{Name} & \textbf{Description} \\
\hline \hline \\
lvl & set the current log level  \\ \hline
\end{tabular}

\subsection{GET\_LOG\_LEVEL}

get current log level as string

\paragraph{Syntax:} \verb|lvl = get_log_level()|

Outputs:

\begin{tabular}{|l|p{5cm}|}
\hline
\textbf{Name} & \textbf{Description} \\
\hline \hline \\
lvl & get the current log level  \\ \hline
\end{tabular}

\subsection{DEBG}

log a debug message

\paragraph{Syntax:} \verb|debg(fmt, ..., [addprefix], [addnewline])|

Inputs:

\begin{tabular}{|l|p{5cm}|}
\hline
\textbf{Name} & \textbf{Description} \\
\hline \hline \\
addprefix & optional boolean to indicate if the prefix should be prepended  \\ \hline
addnewline & optional boolean to indicate if a new line should be appended  \\ \hline
fmt & Message to log. Formatting available  \\ \hline
\end{tabular}

\subsection{ERR}

log a error message

\paragraph{Syntax:} \verb|err(fmt, ..., [addprefix], [addnewline])|

Inputs:

\begin{tabular}{|l|p{5cm}|}
\hline
\textbf{Name} & \textbf{Description} \\
\hline \hline \\
addprefix & optional boolean to indicate if the prefix should be prepended  \\ \hline
addnewline & optional boolean to indicate if a new line should be appended  \\ \hline
fmt & Message to log. Formatting available  \\ \hline
\end{tabular}

\subsection{LOG\_FILE}

gets or sets the log file

\paragraph{Syntax:} \verb|fname = log_file([filename])|

Inputs:

\begin{tabular}{|l|p{5cm}|}
\hline
\textbf{Name} & \textbf{Description} \\
\hline \hline \\
filename & optional filename to set  \\ \hline
\end{tabular}
Outputs:

\begin{tabular}{|l|p{5cm}|}
\hline
\textbf{Name} & \textbf{Description} \\
\hline \hline \\
fname & current logfile  \\ \hline
\end{tabular}

\subsection{LOG\_LEVEL}

gets or sets the log level

\paragraph{Syntax:} \verb|ll = log_level([level])|

Inputs:

\begin{tabular}{|l|p{5cm}|}
\hline
\textbf{Name} & \textbf{Description} \\
\hline \hline \\
level & optional log level to set  \\ \hline
\end{tabular}
Outputs:

\begin{tabular}{|l|p{5cm}|}
\hline
\textbf{Name} & \textbf{Description} \\
\hline \hline \\
ll & current log level  \\ \hline
\end{tabular}

\subsection{WARN}

log a warning message

\paragraph{Syntax:} \verb|warn(fmt, ..., [addprefix], [addnewline])|

Inputs:

\begin{tabular}{|l|p{5cm}|}
\hline
\textbf{Name} & \textbf{Description} \\
\hline \hline \\
addprefix & optional boolean to indicate if the prefix should be prepended  \\ \hline
addnewline & optional boolean to indicate if a new line should be appended  \\ \hline
fmt & Message to log. Formatting available  \\ \hline
\end{tabular}

\subsection{INFO}

log a info message

\paragraph{Syntax:} \verb|info(fmt, ..., [addprefix], [addnewline])|

Inputs:

\begin{tabular}{|l|p{5cm}|}
\hline
\textbf{Name} & \textbf{Description} \\
\hline \hline \\
addprefix & optional boolean to indicate if the prefix should be prepended  \\ \hline
addnewline & optional boolean to indicate if a new line should be appended  \\ \hline
fmt & Message to log. Formatting available  \\ \hline
\end{tabular}

\subsection{LOG\_MSG}

internal logging function

\paragraph{Syntax:} \verb|log_msg(newfmt, fmt, ..., [addprefix], [addnewline])|

Inputs:

\begin{tabular}{|l|p{5cm}|}
\hline
\textbf{Name} & \textbf{Description} \\
\hline \hline \\
newfmt & Prefix format  \\ \hline
addnewline & optional boolean to indicate if a new line should be appended  \\ \hline
fmt & Message to log. Formatting available  \\ \hline
addprefix & optional boolean to indicate if the prefix should be prepended  \\ \hline
\end{tabular}

\subsection{SUCC}

log a success message

\paragraph{Syntax:} \verb|succ(fmt, ..., [addprefix], [addnewline])|

Inputs:

\begin{tabular}{|l|p{5cm}|}
\hline
\textbf{Name} & \textbf{Description} \\
\hline \hline \\
addprefix & optional boolean to indicate if the prefix should be prepended  \\ \hline
addnewline & optional boolean to indicate if a new line should be appended  \\ \hline
fmt & Message to log. Formatting available  \\ \hline
\end{tabular}

\subsection{RECONSTRUCT\_MATRIX\_BY\_OVERWRITE}

reconstructs a given sparse matrix into a full integral image

\paragraph{Syntax:} \verb|outmatrix = reconstruct_matrix_by_overwrite(integral)|

Inputs:

\begin{tabular}{|l|p{5cm}|}
\hline
\textbf{Name} & \textbf{Description} \\
\hline \hline \\
integral & integral struct with fields scores, coords, I\_size  \\ \hline
\end{tabular}
Outputs:

\begin{tabular}{|l|p{5cm}|}
\hline
\textbf{Name} & \textbf{Description} \\
\hline \hline \\
outmatrix & Expanded integral matrix  \\ \hline
\end{tabular}

\subsection{SPARSE\_CODEBOOK\_INTEGRAL}

calculates a codebook entry from a kd-Tee based integral image

\paragraph{Syntax:} \verb|codebook = sparse_codebook_integral(integral, queryX, queryY)|

Inputs:

\begin{tabular}{|l|p{5cm}|}
\hline
\textbf{Name} & \textbf{Description} \\
\hline \hline \\
x, y & x and y coordinates (2 elements per vector)  \\ \hline
integral & integral struct with fields tree.x and tree.y. Scores mustnot be summed up beforehand.  \\ \hline
\end{tabular}
Outputs:

\begin{tabular}{|l|p{5cm}|}
\hline
\textbf{Name} & \textbf{Description} \\
\hline \hline \\
codebook & Codebook vector  \\ \hline
\end{tabular}

\subsection{RECONSTRUCT\_MATRIX}

reconstructs a given sparse matrix into a full integral image

\paragraph{Syntax:} \verb|outmatrix = reconstruct_matrix(inmatrix)|

Inputs:

\begin{tabular}{|l|p{5cm}|}
\hline
\textbf{Name} & \textbf{Description} \\
\hline \hline \\
inmatrix & DxWxH Matrix with changed cells filled, everything else 0  \\ \hline
\end{tabular}
Outputs:

\begin{tabular}{|l|p{5cm}|}
\hline
\textbf{Name} & \textbf{Description} \\
\hline \hline \\
outmatrix & The same matrix but expanded into a integral matrix.Input matrix is reused for speed  \\ \hline
\end{tabular}

\subsection{SPARSE\_CODEBOOK}

calculates a codebook entry from a kd-Tee based integral image

\paragraph{Syntax:} \verb|codebook = sparse_codebook(integral, queryX, queryY)|

Inputs:

\begin{tabular}{|l|p{5cm}|}
\hline
\textbf{Name} & \textbf{Description} \\
\hline \hline \\
queryX, queryY & requested coordinates  \\ \hline
integral & integral struct with fields tree.x and tree.y. Scores mustnot be summed up beforehand.  \\ \hline
\end{tabular}
Outputs:

\begin{tabular}{|l|p{5cm}|}
\hline
\textbf{Name} & \textbf{Description} \\
\hline \hline \\
codebook & Codebook vector  \\ \hline
\end{tabular}

\subsection{RECONSTRUCT\_MATRIX\_BY\_SUM}

reconstructs a given sparse matrix into a full integral image

\paragraph{Syntax:} \verb|outmatrix = reconstruct_matrix_by_sum(integral)|

Inputs:

\begin{tabular}{|l|p{5cm}|}
\hline
\textbf{Name} & \textbf{Description} \\
\hline \hline \\
integral & integral struct with fields scores, coords, I\_size. Scores mustnot be summed up beforehand.  \\ \hline
\end{tabular}
Outputs:

\begin{tabular}{|l|p{5cm}|}
\hline
\textbf{Name} & \textbf{Description} \\
\hline \hline \\
outmatrix & Expanded integral matrix  \\ \hline
\end{tabular}

\subsection{GET\_CURRENT\_SCALES\_BY\_INDEX}

get the current scales by an index

\paragraph{Syntax:} \verb|current_scales = get_current_scales_by_index(si, unique_scales, scale_sizes)|

Inputs:

\begin{tabular}{|l|p{5cm}|}
\hline
\textbf{Name} & \textbf{Description} \\
\hline \hline \\
si & index  \\ \hline
unique\_scales & available scales  \\ \hline
scale\_sizes & amount of scales per split  \\ \hline
\end{tabular}
Outputs:

\begin{tabular}{|l|p{5cm}|}
\hline
\textbf{Name} & \textbf{Description} \\
\hline \hline \\
current\_scales & unique list of scales  \\ \hline
\end{tabular}

\subsection{SORT\_CACHE\_FILES}

sort a list of cache files by sizes extracted by the given format

\paragraph{Syntax:} \verb|[files, sizes] = sort_cache_files(files, format)|

Inputs:

\begin{tabular}{|l|p{5cm}|}
\hline
\textbf{Name} & \textbf{Description} \\
\hline \hline \\
files & list of possible files  \\ \hline
format & format string to extract the sizes  \\ \hline
\end{tabular}
Outputs:

\begin{tabular}{|l|p{5cm}|}
\hline
\textbf{Name} & \textbf{Description} \\
\hline \hline \\
sizes & list of extracted sizes  \\ \hline
files & list of resorted files  \\ \hline
\end{tabular}

\subsection{FILTER\_FEATURE\_BY\_SCALE}

filters given features by a list of of given scales

\paragraph{Syntax:} \verb|filtered = filter_feature_by_scale(current_scales, feature)|

Inputs:

\begin{tabular}{|l|p{5cm}|}
\hline
\textbf{Name} & \textbf{Description} \\
\hline \hline \\
feature & feature struct  \\ \hline
current\_scales & Vector of possible scales  \\ \hline
\end{tabular}
Outputs:

\begin{tabular}{|l|p{5cm}|}
\hline
\textbf{Name} & \textbf{Description} \\
\hline \hline \\
filtered & logic vector  \\ \hline
\end{tabular}

\subsection{GET\_AVAILABLE\_SCALES}

get the available scales by this feature struct

\paragraph{Syntax:} \verb|[unique_scales, scale_sizes] = get_available_scales(params, feature)|

Inputs:

\begin{tabular}{|l|p{5cm}|}
\hline
\textbf{Name} & \textbf{Description} \\
\hline \hline \\
feature & feature struct  \\ \hline
params & Configuration struct  \\ \hline
\end{tabular}
Outputs:

\begin{tabular}{|l|p{5cm}|}
\hline
\textbf{Name} & \textbf{Description} \\
\hline \hline \\
scale\_sizes & amount of scales per split  \\ \hline
unique\_scales & unique list of scales  \\ \hline
\end{tabular}

\subsection{FILTER\_CACHE\_FILES}

filters a list of cache files based on the best matching query size

\paragraph{Syntax:} \verb|filename = filter_cache_files(params, files, sizes, requested_size)|

Inputs:

\begin{tabular}{|l|p{5cm}|}
\hline
\textbf{Name} & \textbf{Description} \\
\hline \hline \\
sizes & corresponding sizes  \\ \hline
files & list of possible files  \\ \hline
params & Configuration struct  \\ \hline
requested\_size & size of query part  \\ \hline
\end{tabular}
Outputs:

\begin{tabular}{|l|p{5cm}|}
\hline
\textbf{Name} & \textbf{Description} \\
\hline \hline \\
filename & path of best matching file  \\ \hline
\end{tabular}

\subsection{GET\_CURRENT\_SCALES\_BY\_SIZE}

get the current scales by a query size

\paragraph{Syntax:} \verb|current_scales = get_current_scales_by_size(params, unique_scales, scale_sizes, roi_size)|

Inputs:

\begin{tabular}{|l|p{5cm}|}
\hline
\textbf{Name} & \textbf{Description} \\
\hline \hline \\
scale\_sizes & amount of scales per split  \\ \hline
roi\_sizes & query size  \\ \hline
params & Configuration struct  \\ \hline
unique\_scales & available scales  \\ \hline
\end{tabular}
Outputs:

\begin{tabular}{|l|p{5cm}|}
\hline
\textbf{Name} & \textbf{Description} \\
\hline \hline \\
current\_scales & unique list of scales  \\ \hline
\end{tabular}

\subsection{GET\_POSSIBLE\_CACHE\_FILES}

get all possible cache files based on a format string

\paragraph{Syntax:} \verb|files = get_possible_cache_files(cachename)|

Inputs:

\begin{tabular}{|l|p{5cm}|}
\hline
\textbf{Name} & \textbf{Description} \\
\hline \hline \\
cachename & format string to search for. Replaces %d by *  \\ \hline
\end{tabular}
Outputs:

\begin{tabular}{|l|p{5cm}|}
\hline
\textbf{Name} & \textbf{Description} \\
\hline \hline \\
files & list of possible files  \\ \hline
\end{tabular}
