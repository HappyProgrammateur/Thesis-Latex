% !TeX encoding = UTF-8
% !TeX root = ../main.tex
% !TeX spellcheck = en_US
\chapter{Introduction}


This thesis discusses one of the main problems in image processing, which is about the retrieval of similar parts from large image databases.
Beside the general problem of scoring the similarity of two images as a whole, the required effort increases if it is required to find only parts inside of images based on a sample image. In contrast to the comparison of full images, there is no information available which parts of an image actually contain objects which could be compared.
Typical solutions try to compare every possible image part (most of the time referred as window) with the sample image to find similar ones.
\bigskip

As some of the current algorithms used by the Computer Vision Group at the \ac{HCI} are using the ExemplarSVM \cite{Malisiewicz2011} framework with a sliding window technique, they are therefore researching for approaches to reduce the amount of windows which have to be scored during a search. 
The final goal of this thesis was to find an algorithm which could be run in beforehand to filter the amount of images or even the windows which have to be searched through the algorithm. It turned out that this task was rather difficult to accomplish if the time needed for the afterwards run of ExemplarSVM algorithm was taken into account. Therefore it was also evaluated if the algorithm could be used for its own or could reduce the images in the image database to small parts so that only a couple of windows per image have to be scored.

The following pages describe the algorithms involved, the experiments and their results and the framework which was developed during the research.
