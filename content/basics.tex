\chapter{Basics}

This chapter covers the terminology and algorithms used during this thesis. 

\section{Terminology}
\label{sec:basic:terminology}

During this thesis, a common terminology regarding image part retrieval is used. The noun \textit{image} $I$ typically defines a whole picture without any further modifications or transformations like cutting, rotating, scaling, normalization or gradient/feature computations. An image \textit{part} on the other hand denotes a small part of an image defined by a rectangle bounding box. Such a bounding box is defined by its coordinate $(x_{tl},y_{tl})$ of the top left corner and the coordinate of the bottom right corner $(x_{br},y_{br})$ or its width and height. A part can consist of several, smaller bounding boxes. Each of those smaller boxes are called \textit{patches} and are represented by a \textit{feature vector}. \textit{Feature vectors} are a comparable way to represent the information in a patch by different factors as for example color distribution, gradients and contours. Typically each dimension of such vectors stands for a defined type of information within the patch. The co-occurrence of multiple patches in an image part is most of the time used to describe the object contained.

If one wants to know if an image contains a specific object, it is typically required to define which patches extracted from the image belong together and therefore could describe the object. The most common approach is \textit{sliding window}. In this approach, a \textit{window} is represented by a (sometimes fuzzy) bounding box which groups the patches together. As it is nearly impossible to test all combinations of patches, the sliding window approach tries to approximate the groupings by defining different window sizes and dimensions and slides them over the image to generate multiple hypothetical groups.

Therefore in this thesis, an image part is equivalent to a bounding box inside of a picture without any further processing. A patch defines an image part which was used to extract a feature vector from and a bounding box enclosing patches is defined as window.
\bigskip

The tests and experiments during this thesis were measured by two scales. The first scale is represented by the \textit{detection rate} or \textit{performance} of the algorithm. The performance is expressed by the precision and recall of the results. This method originally comes from the field of information retrieval. The precision in general denotes the amount of relevant documents which were retrieved compared to all retrieved documents, whereas the recall represents the ratio between the relevant documents retrieved and the total amount of relevant documents.

In terms of the current application, the values can be computed by \ref{eqn:precision_recall}.

\begin{align}
 \text{precision}=\frac{|\{\text{true positives}\}|}{|\{\text{results}\}|} &&&
 \text{recall}=\frac{|\{\text{true positives}\}|}{|\{\text{total expected positives}\}|} 
 \label{eqn:precision_recall}
\end{align}

Another factor which is important to look at is the time required to perform a search for an image part. This was also the main requirement as the original idea for this thesis was to reduce the effort for existing algorithms by performing a lookup beforehand. In the following chapters this will be noted as \textit{execution speed} or \textit{computation speed}.
