\chapter{Current Approaches}

The area of image part retrieval based on the content is a field which is formulated since the 1990s \cite{eakins1999content} \cite{rui1997content} \cite{osuna1997training}. As the general field of image classification is focused on algorithms to provide the most accurate classification, the part retrieval also has a higher dependency on the processing time. This circumstance is based on the problem that current segmentation and grouping techniques provide not as good results as compared to a the classification of multiple image regions in different scales and ratios \cite{book:848523}. As the processing speed becomes more important if more regions are searched, the initial approaches just used the color distribution to define similarities to get as fast as possible search times. The structural and texture information come into play as the computation power becomes more available and allowed to use more expensive techniques. Many approaches were designed for a specific task to get the optimum ratio between processing time and query results. One of the examples is the face detection algorithm based on haar features and boosting cascades by Viola and Jones \cite{viola2001rapid}. Current approaches are mostly based on feature describing algorithms already used in content based image classification tasks. Some of the most used algorithms are \acf{SIFT} \cite{Lowe2004}, \acf{HOG} \cite{Dalal2005} and \acf{SURF} \cite{bay2008speeded}. As those feature descriptors already produced reasonable results for searching a particular object with classification methods like \aclp{SVM} \cite{cortes1995support}, the research goes into building up visual categories to transfer meta information between the different detections.

One major algorithm is the ExemplarSVM \cite{Malisiewicz2011}. It provided an approach to find objects contained in images not only based on the query objects type, but also to categories them into different orientations and shapes.
It is based on the \ac{HOG} feature descriptors and \ac{SVM} for the classification. As the training of an appropriate \ac{SVM} for an object category required a large set of negative samples to be used together with multiple positive samples to get a adequate generic classifier, Malisiewicz et al. used the approach to train multiple \acp{SVM} with lesser samples. Each of this classifiers are trained with only one positive sample and multiple negative samples. This allowed them to use one \ac{SVM} only for visually similar objects instead of a complete category and therefore use a \ac{SVM} algorithm which only has to solve this simpler task and can compete against an over-fitting based on the single positive sample.%TODO