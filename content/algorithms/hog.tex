\section{\acf{HOG}}
\label{sec:hog}

\acs{HOG} were first introduced by Dalal and Triggs in 2005 \cite{Dalal2005}. Extracting \acs{HOG} from images consist of computing the gradients of pixel cells, grouped in blocks (the actual descriptors). The authors evaluated the usage of gamma/color normalization, but got only modest effects on performance. Gray scaling the images even reduces the performance.

The first step for extracting \ac{HOG} from images is therefore the computation of gradients. It was evaluated that simple 1-D $[-1, 0, 1]$ masks with a $\sigma = 0$ smoothing scale produces the best results. Color channels are computed separately and the one with the largest norm is used as the pixel's gradient vector.

The second step handles the binning by creating the cell histograms. In this step, each pixel within a $8 \times 8$ cell does a weighted vote for a histogram channel based on his gradient orientation. The weight of the vote is calculated by the magnitude of the gradient.

As the gradient strengths vary over a wide range, based on illumination variations and foreground-background contrast, a local contrast normalization is required as a third step. This is done by grouping the histogram cells into larger blocks. The authors evaluated two grouping approaches, rectangular HOG (\acs{R-HOG}) and circular HOG (\acs{C-HOG}). \acs{R-HOG} are typical organized as $3\times3$ or $5\times5$ cell blocks. \acs{C-HOG} are grouped by a log-polar grid. According to the authors, at least two radial bins and four angular bins (makes a total of five to eight bins, depending if the center is also divided) are needed for good performance.
For the normalization multiple approaches produced the same performance:
\begin{itemize}
	\item L2-Hys $v \rightarrow \frac{\max(v / \sqrt{||v||_2^2 + \epsilon^2}, 0.2)}{\sqrt{||\max(v / \sqrt{||v||_2^2 + \epsilon^2}, 0.2)||_2^2 + \epsilon^2}}$ %TODO renormalizing
	\item L2-norm $v \rightarrow v / \sqrt{||v||_2^2 + \epsilon^2}$
\end{itemize}
The L1-sqrt ($v \rightarrow \sqrt{v / ||v||_1 + \epsilon}$) reduced the performance by 5\%, whilst omitting the normalization reduced it by 27\% according to \cite{Dalal2005}.