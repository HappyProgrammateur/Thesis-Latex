%\begin{table}
\begin{longtable}{%
|%
p{0.2\textwidth}%
|%
>{\raggedright\arraybackslash}%
p{0.135\textwidth}%
|%
>{\raggedright\arraybackslash}%
p{0.135\textwidth}%
|%
>{\raggedright\arraybackslash}%
p{0.4\textwidth}%
|%
}
\hline \textbf{Parameter} & \textbf{Possible values} & \textbf{Default} & \textbf{Description} \\ 
\hline
\hline class & A PASCAL class & bicycle & Specifies the class to restrict the bounding boxes to \\ 
\hline parts & $x \in \mathbb{N} \textbackslash 0$ & 4 & Number of parts per window \\ 
\hline clusters & $x \in \mathbb{N} \textbackslash 0$ & 1000 & Number of k-Means clusters to use \\ 
\hline integrals\allowbreak\_scale\allowbreak\_factor & $x \in \mathbb{R}, 0 < x \le 1$ & 1 & Specifies the percentage to which a integral image should be downsampled to \\ 
\hline stream\allowbreak\_max & $x \in \mathbb{N} \textbackslash 0$ & 100 & Maximum amount of images used from a PASCAL stream \\ 
\hline stream\allowbreak\_name & Any string & query & Name of the PASCAL stream to read images from \\ 
\hline codebook\allowbreak\_type & double, single & double & Internal datatype of the codebooks \\ 
\hline codebook\allowbreak\_scales\allowbreak\_count & $x \in \mathbb{N} \textbackslash 0$ & 3 & Amount of scale ranges in which integral images will be split into \\ 
\hline nonmax\allowbreak\_type\allowbreak\_min & true, false & true & Specifies if $\frac{\text{area}(A \cap B)}{\min(\text{area}(A), \text{area}(B))}$ or $\frac{\text{area}(A \cap B)}{\text{area}(A \cup B)}$ should be used for a non-maximum suppression \\ 
\hline use\allowbreak\_calibration & true, false & true & If true, a score calibration based on the query image is performed \\ 
\hline features\allowbreak\_per\allowbreak\_roi & $x \in \mathbb{R}, 0 < x$ & 2 & Amount of feature patches which have to exist per window axis \\ 
\hline query\allowbreak\_from\allowbreak\_integral & true, false & false & If true, the query codebooks will be retrieved from the image database instead of extracted from the query image \\ 
\hline default\allowbreak\_query\allowbreak\_file & An image id & 2008\allowbreak\_004363 & Used to automate the experiments instead of asking interactively for a query image \\ 
\hline default\allowbreak\_bounding\allowbreak\_box & [x, y, width, height] & Extracted from the PASCAL dataset & The bounding box of the query image part \\ 
\hline use\allowbreak\_libsvm\allowbreak\_classification & true, false & true & If true, the \verb|svmpredict| function is used to classify the database codebooks. Otherwise a custom implementation is used \\ 
\hline expand\allowbreak\_bboxes & true, false & true & Expands the bounding boxes by $\frac{1}{2}$ of the average feature patch size \\ 
\hline naiive\allowbreak\_integral\allowbreak\_backend & true, false & true & Specifies to use full integral images without any memory optimization techniques \\ 
\hline window\allowbreak\_margin & $x \in \mathbb{N} \textbackslash 0$ & 10 & Amount of pixels each window is shifted during the sliding window generation  \\ 
\hline max\allowbreak\_window\allowbreak\_scales & $x \in \mathbb{N} \textbackslash 0$ & 10 & Maximum amount of times a scaling is performed during the sliding window generation \\ 
\hline min\allowbreak\_window\allowbreak\_size & $x \in \mathbb{R}, 0 < x \le 1$ & 0.5 & Required percentage of a window area which has to be placed over an image (windows are getting clipped at the image boundaries)\\
\hline window\allowbreak\_generation\allowbreak\_relative\allowbreak\_move & $x \in \mathbb{R}$ & 0 & Percentage of its size a window should move during sliding window (0 means fixed window\allowbreak\_margin pixels) \\
\hline max\allowbreak\_window\allowbreak\_image\allowbreak\_ratio & $x \in \mathbb{R}, 0 < x \le 1$ & 1.0 & Maximum allowed size of a window in relation to the image \\
\hline use\allowbreak\_kdtree & true, false & false & Enables the kd-Tree storage backend \\ 
\hline integral\allowbreak\_backend\allowbreak\_overwrite & true, false & false & Enables the integral matrix reconstruction via overwriting the bottom right values \\ 
\hline integral\allowbreak\_backend\allowbreak\_sum & true, false & false & Enables the integral matrix reconstruction via summing up the bottom right values \\ 
\hline integral\allowbreak\_backend\allowbreak\_matlab\allowbreak\_sparse & true, false & false & Enables the \MATLAB sparse storage backend \\ 
\hline precalced\allowbreak\_windows & true, false & false & Disables integral images and precalculates codebooks in a sliding window manner \\ 
\hline inverse\allowbreak\_search & true, false & false & Use the information of integral images to reduce the amount of windows generated. \\
\hline memory\allowbreak\_cache & true, false & true & Reuse already loaded databases and negative codebooks (useful for development) \\
\hline use\allowbreak\_threading & true, false & true & Enables the use of the parallel toolbox \\
\hline window\allowbreak\_prefilter & true, false & false & Prefilter the windows even further to reduce the amount of searched windows \\
\hline fisher\allowbreak\_backend & true, false & false & Use fisher vectors instead of k-means clusters (experimental as it is not practical) \\
\hline
\caption{Configuration parameters for the part retrieval framework}
\label{tab:configuration_framework}
\end{longtable}
%\end{table}
