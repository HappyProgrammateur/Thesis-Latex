Die Suche in großen Bilddatenbanken ist ein zentrales Problemfeld in der Computer Vision. Das Feld wird schon seit Jahrzehnten intensiv erforscht und hat verschiedenste Algorithmen hervorgebracht welche immer schneller, präzisere Ergebnisse liefern. Im Rahmen der Thesis wurde nach Optimierungen für die Suche nach Bildausschnitten in Bildsammlungen geforscht und versucht durch allgemeine Vorberechnungen den Vorgang zu beschleunigen. Als Referenz für die Implementierungen wurde das ExemplarSVM-Framework \cite{Malisiewicz2011} aufgrund seiner weiten Verbreitung verwendet. Die Ergebnisse der verschiedenen Ansätze zeigen dass zwar noch Verbesserungen notwendig sind um mit aktuellen Algorithmen vergleichbar zu sein, jedoch auch das Potential für die Beschleunigung der Suchen gegeben sind.