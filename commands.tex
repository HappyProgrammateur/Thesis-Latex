\newcommand{\MATLAB}{MATLAB\textsuperscript{\textregistered}~}
\newcommand{\eqnref}[1]{equation \ref{eqn:#1}}
\newcommand{\figref}[1]{figure \ref{fig:#1}}
\newcommand{\tabref}[1]{table \ref{tab:#1}}
\newcommand{\lstref}[1]{listing \ref{lst:#1}}
\newcommand{\algref}[1]{algorithm \ref{alg:#1}}
\newcommand{\Eqnref}[1]{Equation \ref{eqn:#1}}
\newcommand{\Figref}[1]{Figure \ref{fig:#1}}
\newcommand{\Tabref}[1]{Table \ref{tab:#1}}
\newcommand{\Lstref}[1]{Listing \ref{lst:#1}}
\newcommand{\Algref}[1]{Algorithm \ref{alg:#1}}

\mathchardef\breakingcomma\mathcode`\,
{\catcode`,=\active
  \gdef,{\breakingcomma\discretionary{}{}{}}
}
\newcommand{\mathlist}[1]{$\{\mathcode`\,=\string"8000 #1\}$}

\makeatletter
\newcommand\footnoteref[1]{\protected@xdef\@thefnmark{\ref{#1}}\@footnotemark}
\makeatother