% ##################################################
% Unterstuetzung fuer die deutsche Sprache
% ##################################################
%\usepackage{ngerman}
%\usepackage[ngerman]{babel}

% ##################################################
% Dokumentvariablen
% ##################################################

% Persoenliche Daten
\newcommand{\docNachname}{Schmid}
\newcommand{\docVorname}{Timo}
\newcommand{\docStrasse}{In der Neckarhelle 99}
\newcommand{\docOrt}{Heidelberg}
\newcommand{\docPlz}{69118}
\newcommand{\docEmail}{timo.schmid@stud.uni-heidelberg.de}
\newcommand{\docMatrikelnummer}{3169575}
\newcommand{\docName}{\docVorname~\docNachname}

% Dokumentdaten
\newcommand{\docTitle}{Efficient Large Scale Part Retrieval}
%\newcommand{\docUntertitle}{} % Kein Untertitel
\newcommand{\docUntertitle}{?}
% Arten der Arbeit: Bachelorthesis, Masterthesis, Seminararbeit, Diplomarbeit
\newcommand{\docArtDerArbeit}{Masterthesis}
%Studiengaenge: Allgemeine Informatik Bachelor, Computer Networking Bachelor,
% Software-Produktmanagement Bachelor, Advanced Computer Science Master
\newcommand{\docStudiengang}{Applied Computer Science}
\newcommand{\docAbgabedatum}{31.10.2015}
\newcommand{\docErsterReferent}{Prof. Dr. Björn Ommer}
\newcommand{\docZweiterReferent}{Masato Takami}

% ##################################################
% Allgemeine Pakete
% ##################################################

% Mathematische Formeln und Symbole
\usepackage{amsmath}

% Bearbeiten von floatings
\usepackage{floatrow}

% Abbildungen einbinden
\usepackage{graphicx}

\usepackage{caption}

% Abbildungen gruppieren
\usepackage{subcaption}

% Zusaetsliche Sonderzeichen
\usepackage{dingbat}

% Abkuerzungsverzeichnis
\usepackage{acronym}

% Farben
\usepackage{color}
\usepackage[usenames,dvipsnames,svgnames,table]{xcolor}

% Maskierung von URLs und Dateipfaden
\usepackage{url}

% Deutsche Anfuehrungszeichen
%\usepackage[babel, german=quotes]{csquotes}

% Pakte zur Index-Erstellung (Schlagwortverzeichnis)
\usepackage{index}
\makeindex

% appendices
\usepackage[titletoc]{appendix}

% Ipsum Lorem
% Paket wird nur für das Beispiel gebraucht und kann gelöscht werden
%\usepackage{lipsum}

% Pseudocode
%\usepackage[boxed]{algorithm}
%\usepackage{algpseudocode}

% ##################################################
% Seitenformatierung
% ##################################################
\usepackage[
	portrait,
	bindingoffset=1.5cm,
	inner=2.5cm,
	outer=2.5cm,
	top=3cm,
	bottom=2cm,
	includeheadfoot
	]{geometry}

% ##################################################
% Kopf- und Fusszeile
% ##################################################

\usepackage{fancyhdr}

\pagestyle{fancy}
\fancyhf{}
\fancyhead[EL,OR]{\sffamily\thepage}
\fancyhead[ER,OL]{\sffamily\leftmark}

\fancypagestyle{headings}{}

\fancypagestyle{plain}{}

\fancypagestyle{empty}{
  \fancyhf{}
  \renewcommand{\headrulewidth}{0pt}
}

\newcommand{\runningauthor}{}
%Kein "Kapitel # NAME" in der Kopfzeile
\renewcommand{\chaptermark}[1]{%
	\markboth{#1}{}%
   	\markboth{\thechapter.\ #1\runningauthor%
   	}{}%
}

% ##################################################
% Schriften
% ##################################################

% Stdandardschrift festlegen
\renewcommand{\familydefault}{\sfdefault}

% Standard Zeilenabstand: 1,5 zeilig
\usepackage{setspace}
\onehalfspacing 

% Schriftgroessen festlegen
\addtokomafont{chapter}{\sffamily\large\bfseries} 
\addtokomafont{section}{\sffamily\normalsize\bfseries} 
\addtokomafont{subsection}{\sffamily\normalsize\mdseries} 
\addtokomafont{subsubsection}{\sffamily\normalsize\mdseries} 
\addtokomafont{caption}{\sffamily\normalsize\mdseries} 

\usepackage{pifont}% http://ctan.org/pkg/pifont
\newcommand{\cmark}{\ding{51}}%
\newcommand{\xmark}{\ding{55}}%

% ##################################################
% Referenzen innerhalb des Doks (z.B. Abb. XY)
% ##################################################
\usepackage{prettyref}
%%% Für Kapitel %%%
\newrefformat{cha}{Kapitel~\ref{#1} ab Seite \pageref{#1}}
%%% Für Abschnitte %%%
\newrefformat{sec}{Abschnitt~\ref{#1} auf Seite \pageref{#1}}
%%% Für Abbildungen %%%
\newrefformat{fig}{Abbildung~\ref{#1} auf Seite \pageref{#1}}
%%% Für Tabellen %%%
\newrefformat{tab}{Tabelle~\ref{#1} auf Seite \pageref{#1}} 
%%% Für Listings %%%
\newrefformat{lst}{Listing~\ref{#1} auf Seite \pageref{#1}} 
%%% Für Algorithmen %%%
\newrefformat{alg}{Algorithmus~\ref{#1} auf Seite \pageref{#1}} 

% ##################################################
% Inhaltsverzeichnis / Allgemeine Verzeichniseinstellungen
% ##################################################

\usepackage{tocloft}

% Punkte auch bei Kapiteln
\renewcommand{\cftchapdotsep}{3}
\renewcommand{\cftdotsep}{3}

% Schriftart und -groesse im Inhaltsverzeichnis anpassen
\renewcommand{\cftchapfont}{\sffamily\normalsize}
\renewcommand{\cftsecfont}{\sffamily\normalsize}
\renewcommand{\cftsubsecfont}{\sffamily\normalsize}
\renewcommand{\cftchappagefont}{\sffamily\normalsize}
\renewcommand{\cftsecpagefont}{\sffamily\normalsize}
\renewcommand{\cftsubsecpagefont}{\sffamily\normalsize}

%Zeilenabstand in den Verzeichnissen einstellen
\setlength{\cftparskip}{.5\baselineskip}

% ##################################################
% Abbildungsverzeichnis und Abbildungen
% ##################################################

\usepackage{caption}
\usepackage{chngcntr}
\usepackage{wrapfig}

% Nummerierung von Abbildungen
\renewcommand{\thefigure}{\arabic{figure}}

\counterwithout{figure}{chapter}

% Abbildungsverzeichnis anpassen
%\renewcommand{\cftfigpresnum}{Abbildung }
\renewcommand{\cftfigaftersnum}{:}

% Breite des Nummerierungsbereiches [Abbildung 1:]
\newlength{\figureLength}
\settowidth{\figureLength}{\bfseries\cftfigpresnum\cftfigaftersnum}
\setlength{\cftfignumwidth}{\figureLength}
\setlength{\cftfigindent}{0cm}

% Schriftart anpassen
\renewcommand\cftfigfont{\sffamily}
\renewcommand\cftfigpagefont{\sffamily}

% ##################################################
% Tabellenverzeichnis und Tabellen
% ##################################################

% Nummerierung von Tabellen
\renewcommand{\thetable}{\arabic{table}}

\counterwithout{table}{chapter}

% Tabellenverzeichnis anpassen
%\renewcommand{\cfttabpresnum}{Tabelle }
\renewcommand{\cfttabaftersnum}{:}

% Breite des Nummerierungsbereiches [Abbildung 1:]
\newlength{\tableLength}
\settowidth{\tableLength}{\bfseries\cfttabpresnum\cfttabaftersnum}
\setlength{\cfttabnumwidth}{\tableLength}
\setlength{\cfttabindent}{0cm}

%Schriftart anpassen
\renewcommand\cfttabfont{\sffamily}
\renewcommand\cfttabpagefont{\sffamily}

% Unterdrueckung von vertikalen Linien
\usepackage{booktabs}

% ##################################################
% Listings (Quellcode)
% ##################################################

\usepackage{listings}
\lstset{
	language=java,
	backgroundcolor=\color{white},
	breaklines=true,
	prebreak={\carriagereturn},
 	breakautoindent=true,
 	numbers=left,
 	numberstyle=\tiny,
% 	stepnumber=2,
 	numbersep=5pt,
 	keywordstyle=\color{blue},
   	commentstyle=\color{green},   
   	stringstyle=\color{gray}
}
  	
% ##################################################
% Theoreme
% ##################################################
  	
% Umgebung fuer Beispiele
\newtheorem{beispiel}{Beispiel}

% Umgebung fuer These
\newtheorem{these}{These}

% Umgebung fuer Definitionen
\newtheorem{definition}{Definition}
  	
% ##################################################
% Literaturverzeichnis
% ##################################################

%\usepackage{bibgerm}
\usepackage[backend=bibtex,style=alphabetic]{biblatex}
\addbibresource{bibtex/library.bib}

% ##################################################
% PDF / Dokumenteninternelinks
% ##################################################

\usepackage[
	colorlinks=true,
  	linkcolor=red,
   	citecolor=green,
  	filecolor=magenta,
	urlcolor=cyan,
    bookmarks=true,
    bookmarksopen=true,
    bookmarksopenlevel=3,
    bookmarksnumbered,
    plainpages=false,
    pdfpagelabels=true,
    hyperfootnotes,
    pdftitle ={\docTitle},
    pdfauthor={\docName},
    pdfcreator={\docName},
    hidelinks=true %set true for hiding link colors and borders
    ]{hyperref}
    
% ##################################################
% Glossary
% ##################################################
%\usepackage[xindy]{glossaries}
\usepackage{glossaries}
\makeglossaries 


% ##################################################
% Definitionen
% ##################################################
\def\SymbReg{\textsuperscript{\textregistered}}


% ##################################################
% Layout
% ##################################################
\floatname{algorithm}{Algorithmus}
\floatsetup[algorithm]{capposition=top}
\floatsetup[table]{capposition=top}
\floatsetup[figure]{capposition=top}


\newcommand{\MATLAB}{MATLAB\textsuperscript{\textregistered}~}
\newcommand{\eqnref}[1]{equation \ref{eqn:#1}}
\newcommand{\figref}[1]{figure \ref{fig:#1}}
\newcommand{\tabref}[1]{table \ref{tab:#1}}
\newcommand{\lstref}[1]{listing \ref{lst:#1}}
\newcommand{\algref}[1]{algorithm \ref{alg:#1}}
\newcommand{\Eqnref}[1]{Equation \ref{eqn:#1}}
\newcommand{\Figref}[1]{Figure \ref{fig:#1}}
\newcommand{\Tabref}[1]{Table \ref{tab:#1}}
\newcommand{\Lstref}[1]{Listing \ref{lst:#1}}
\newcommand{\Algref}[1]{Algorithm \ref{alg:#1}}

\mathchardef\breakingcomma\mathcode`\,
{\catcode`,=\active
  \gdef,{\breakingcomma\discretionary{}{}{}}
}
\newcommand{\mathlist}[1]{$\{\mathcode`\,=\string"8000 #1\}$}

\makeatletter
\newcommand\footnoteref[1]{\protected@xdef\@thefnmark{\ref{#1}}\@footnotemark}
\makeatother